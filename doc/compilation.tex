\chapter{Compiling and installing \fcs}
\label{cha:compiling}

\index{Compilation}
\index{Installation}
\index{Configure}

The \fcs library uses the GNU build system and thus compiling and installing consists of the common steps \texttt{configure} and \texttt{make}.
In the following, the steps for building the library from source are explained in more details.


\begin{enumerate}

  \item[1)]
Obtain a copy of the \fcs library sources.
Single releases of the library sources can be downloaded from the official web site (\url{http://www.scafacos.de}).
Furthermore, the sources of the library are hosted on GitHub where the lasted version is publicly available (see \url{http://github.com/scafacos}).
A local copy of the library repository can be obtain with the following command:
\begin{alltt}
  git clone git@github.com:scafacos/scafacos.git
\end{alltt}

Building the library from the git repository requires recent versions of the GNU Autotools, \ie m4, Autoconf, Automake, Libtool.
Details about the specific versions required can be found in Sect.~\ref{cha:buildsystem}.
Executing Autotools has to be performed with the \texttt{bootstrap} script:
\begin{alltt}
  ./bootstrap
\end{alltt}
After executing this script, the source directory should contain a newly generated \texttt{configure} script.

  \item[2)]
Compiling the library consists of the two steps \texttt{configure} and \texttt{make}.
For the following description it is assumed that the library sources are stored in directory \texttt{scafacos/}.

  \begin{enumerate}
  \item[a)]
The library can be build either inside or outside the source tree.
To build the library outside the source tree, change the working directory to the desired build directory:
\begin{alltt}
  mkdir ../build_fcs
  cd ../build_fcs
\end{alltt}

Execute the \texttt{configure} script as follows:
\begin{alltt}
  ../scafacos/configure <options>
\end{alltt}

See \texttt{'./configure --help'} (or \texttt{'./configure --help=recursive'}) to display all supported options.
The main options of the library are listed in Table~\ref{tbl:fcs_options}.

  \item[b)]
Run \texttt{make} to build the library.
Use option \texttt{-j <N>} to execute (at most) \texttt{N} build jobs in parallel.
  \end{enumerate}

  \item[3)] Run test programs:
  \begin{enumerate}
    \item[a)]
Run \texttt{make check} to execute all test programs of the library.
The build system tries to finding out how MPI jobs can be started.
Edit file \texttt{test/defs.in} to modify the method that was automatically determined.
Alternatively, \texttt{configure} option \texttt{MPIEXEC=...} can be used to define the command for executing MPI programs.

    \item[b)]
The test programs are located in \texttt{test/c} and \texttt{test/fortran}.
See scripts \texttt{start\_....sh} to execute single test programs.

    \item[c)]
If the execution of MPI programs with the previous methods fails, then it is also possible to start the test programs manually.
Use scripts \texttt{start\_....sh} (in \texttt{test/c/} or \texttt{test/fortran/}) and replace command \texttt{start\_mpi\_job} with an appropriate command for executing MPI programs.
  \end{enumerate}

  \item[4)]
Run \texttt{make install} to install the header and library files (\ie, to \texttt{/usr/local/} if not specified otherwise with \texttt{--prefix=...} during \texttt{configure}).
Alternatively, the header and library files can be found in \texttt{src/} and \texttt{lib/}, respectively.

  \item[5)]
A configuration file \texttt{scafacos-fcs.pc} for \texttt{pkg-config} is created in \texttt{packages/} and installed with \texttt{make install}.
Use the following commands to determine the compiler and linker flags required to integrate the library into an application program:
\begin{alltt}
  pkg-config --cflags scafacos-fcs
  pkg-config --libs scafacos-fcs
\end{alltt}

\end{enumerate}

\begin{table}
\begin{tabular}{|p{0.4\textwidth}|p{0.6\textwidth}|}
\hline
\texttt{-C} & Enable caching to speed up \texttt{configure}. \\
\hline
\texttt{--prefix=PREFIX} & Install library files to directory \texttt{PREFIX}. \\
\hline
\texttt{--enable-fcs-solvers=...} & Enable only the specified list of solvers. \\
\texttt{--disable-fcs-SOLVER} & Disable the specified solver \texttt{SOLVER}. \\
\hline
\texttt{--enable-fcs-int=...} & Set \fcs integer type to the given C type. \\
\texttt{--enable-fcs-float=...} & Set \fcs float type to the given C type. \\
\texttt{--enable-fcs-integer=...} & Set \fcs integer type to the given Fortran type. \\
\texttt{--enable-fcs-real=...} & Set \fcs float type to the given Fortran type. \\
\hline
\texttt{--enable-fcs-info} & Enable info output. \\
\texttt{--enable-fcs-debug} & Enable debug output. \\
\texttt{--enable-fcs-timing} & Enable timing output. \\
\hline
\texttt{--enable-single-lib} & Build and install only a single file \texttt{libfcs.a}. \\
\hline
\texttt{MPICC=...  CFLAGS=...} & Set (MPI) C compiler and flags. \\
\texttt{MPICXX=... CXXFLAGS=...} & Set (MPI) C++ compiler. \\
\texttt{MPIFC=...  FCFLAGS=...} & Set (MPI) Fortran compiler. \\
\hline
\end{tabular}
\caption{List of common \texttt{configure} options of the \fcs library.}
\label{tbl:fcs_options}
\end{table}


%\section*{Additional information for selected platforms}
%
%Configure setups for common systems
%===================================
%
%Generally, if you set both MPICC and CC, or MPICXX and CXX, or MPIFC and FC,
%you may need to ensure that compatible compiler drivers are used.
%
%You may want to add -C to cache configure test results; this speeds up the
%recursive configure scripts.
%
%GNU/Linux
%---------
%It should work to use one of
%
%  ./configure
%  ./configure FCFLAGS=-fno-underscoring
%
%depending on which ABI you need to maintain with respect to other Fortran code.
%If you are using gfortran, you need version 4.3 or newer for Fortran 2003 support.
%
%BlueGene
%--------
%The following alternative configurations should work (the --build and --host
%arguments ensure configure enables cross-compilation mode):
%
%  ./configure --build=powerpc64-bgp-linux-gnu --host=powerpc-ibm-linux \
%     MPICC=mpixlc_r MPIFC=mpixlf2003_r MPICXX=mpixlcxx_r CFLAGS='-qfullpath -O2' \
%     FCFLAGS='-qfullpath -qarch=450 -qtune=450 -qxlf2003=nostopexcept -O3 -qipa -qipa=inline=key2addr'
%
%  ./configure --build=powerpc64-bgp-linux-gnu --host=powerpc-ibm-linux \
%     MPICC=mpixlc_r MPIFC=mpixlf2003_r MPICXX=mpixlcxx_r \
%     CFLAGS='-g -qfullpath -qdbxextra -qcheck' FCFLAGS='-g -qfullpath'
%
%  ./configure --build=powerpc64-bgp-linux-gnu --host=powerpc-ibm-linux \
%     MPICC=mpixlc_r MPICXX=mpixlcxx_r MPIFC=mpixlf2003_r \
%     CPPFLAGS='-I/bgsys/drivers/ppcfloor/comm/include -I/bgsys/drivers/ppcfloor/arch/include' \
%     CFLAGS='-O3 -g -qmaxmem=-1 -qarch=450 -qtune=450' \
%     FCFLAGS='-O3 -g -qmaxmem=-1 -I/bgsys/drivers/ppcfloor/include -qarch=450 -qtune=450' \
%     LDFLAGS='-L/bgsys/drivers/ppcfloor/lib -L/bgsys/local/lapack/lib -L/bgsys/local/lib' \
%     BLAS_LIBS='-lesslbg -lxlf90_r'
%
%Juropa
%------
%
%Building with the Intel compiler suite should work:
%
%  ./configure MPICC=mpicc MPICXX=mpicxx MPIFC=mpif90
