\chapter{Implementation}
\label{cha:implementation}

\index{Implementation}

\section{Licenses}
\label{sec:licenses}

\subsection{Applying the (L)GPL}

To apply the (L)GPL to your program, do the following:

\begin{itemize}
\item Add the following header text in a comment to all of your files
  (source files and text files!):

\begin{verbatim}
  Copyright (C) 2011 <authors>

  This file is part of ScaFaCoS.

  ScaFaCoS is free software: you can redistribute it and/or modify it
  under the terms of the GNU Lesser Public License as published by the
  Free Software Foundation, either version 3 of the License, or (at
  your option) any later version.

  ScaFaCoS is distributed in the hope that it will be useful, but
  WITHOUT ANY WARRANTY; without even the implied warranty of
  MERCHANTABILITY or FITNESS FOR A PARTICULAR PURPOSE.  See the GNU
  Lesser Public License for more details.

  You should have received a copy of the GNU Lesser Public License
  along with this program.  If not, see
  <http://www.gnu.org/licenses/>.
\end{verbatim}

\item Replace the word \texttt{<authors>} with the names of the
  authors that have written the file.  If you want to use the GPL
  instead of the LGPL, replace the word \textit{Lesser} in the text by
  \textit{General} in all three places.
\end{itemize}

\subsection{Simple files}

For simple files, you might prefer not to use the long header of the
(L)GPL, instead you can use:

\begin{verbatim}
Copyright (C) 2011 <authors>

Copying and distribution of this file, with or without modification,
are permitted in any medium without royalty provided the copyright
notice and this notice are preserved.  This file is offered as-is,
without any warranty.
\end{verbatim}
