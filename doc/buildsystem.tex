\chapter{Build System}
\label{cha:buildsystem}
\index{Build System}

\begin{verbatim}
Prerequisites
-------------

This source tree uses GNU autotools
<http://www.gnu.org/software/automake/manual/html_node/Autotools-Introduction.html>

In order to do development, please make sure you have reasonably recent
versions of the following tools installed and in your $PATH:
  GNU m4       >= 1.4.13  http://ftp.gnu.org/gnu/m4/m4-1.4.16.tar.gz
  GNU Autoconf >= 2.64    http://ftp.gnu.org/gnu/autoconf/autoconf-2.68.tar.gz
  GNU Automake >= 1.11    http://ftp.gnu.org/gnu/automake/automake-1.11.3.tar.gz
  GNU Libtool  >= 2.2.6b  http://ftp.gnu.org/gnu/libtool/libtool-2.4.2.tar.gz

Then, set up the autotools infrastructure:
  ./bootstrap

This should create configure, Makefile.in, and config.h.in files.

If that went without trouble, continue as described in the README file.
\end{verbatim}

To learn how to install the Autotools, have a look at
\url{http://www.gnu.org/software/automake/faq/autotools-faq.html#How-do-I-install-the-Autotools-_0028as-user_0029_003f}.

\begin{verbatim}
Build system layout
-------------------

There is a toplevel configure.ac script and helper macros in the m4/
subdirectory.  Each directory where stuff needs to be done later gets
a Makefile.am file which is processed by automake and later by configure.

In order to facilitate modularity of the different solver methods, each
method gets its own configure.ac script in libs/$method/ as well.  The
toplevel configure will call each of the lower-level ones which are
enabled and present in turn.

The generated makefiles support all of the standard targets described here:
<http://www.gnu.org/software/automake/manual/html_node/Standard-Targets.html>


Generated files
---------------

The autotools build system uses several generated files and a few helper
scripts:

1) Helper scripts installed below build-aux/ subdirectories:
- depcomp
- install-sh
- compile
- missing

2a) File generated at bootstrap time, and distributed to end-users:
- aclocal.m4      macro files generated by aclocal
- configure       scripts are generated from configure.ac files and additional
                  macro files (aclocal.m4 and files in some m4/ subdirectory)
- Makefile.in     template files generated by automake (and later converted to
                  Makefile files by config.status)
- config.h.in     header template file generated by autoheader (and later
                  converted to config.h by config.status)

2b) File generated at bootstrap time, NOT distributed to end-users:
- autom4te.cache  directory containing autotools-internal cache files.
                  This may be safely removed at any time.

3) Files generated at configure run time by the end-user:
- config.log      a log file containing detailed configure test results
- config.status   a script file containing the test results
- config.cache    a cache of test results (used when --config-cache is given)
- Makefile        the actual makefiles generated by config.status
- config.h        a project-specific header that should not be installed
- .deps/*         dependency tracking makefile snippets
- stamp-*          makefile stamp files

and of course object files and programs etc.

None of these files should be committed to SVN, because the files in (1) and
in (2) may differ between different autotools versions (causing spurious
differences with "svn diff" when two developers use different versions)
and because the files in (3) are system-dependent.

For further reading see:
<http://www.gnu.org/software/autoconf/manual/html_node/Making-configure-Scripts.html>
<http://www.gnu.org/software/automake/manual/html_node/CVS.html>

How to add a new solver
-----------------------

If you are adding a new solver to the build system, please adjust the
following build system files:

- Either convert your build system below lib/SOLVER to autotools or ensure
  that the usual GNU targets are supported by your makefiles and that VPATH
  building works; see here for more information:
  <http://www.gnu.org/software/automake/manual/html_node/Third_002dParty-Makefiles.html>
  <http://www.gnu.org/software/automake/manual/html_node/Standard-Targets.html>

- Add your solver to the toplevel all_solver_methods macro in the toplevel
  configure.ac script.  If your solver contains GPL code, or relies on GPL
  libraries, add your solver to the gpl_solver_methods macro in the toplevel
  configure.ac script.  Add the list of libraries the user should link to
  to the SCAFACOS_LIBS variable near the end of the script.

- Document in the toplevel README any required libraries for your solver.

- For each Makefile.am you add,
  - adjust the SUBDIRS line in the next-higher Makefile.am file,
  - add an 'AC_CONFIG_FILES([.../Makefile])' line to the next-higher
    configure.ac, with the correct relative path.

- In each Makefile.am file that deals with preprocessed Fortran, add
    include $(top_srcdir)/build-aux/fortran-rules.am

  to ensure .f90 files are preprocessed.

- For each C, C++ source file, add
    #ifdef HAVE_CONFIG_H
    #include <config.h>
    #endif

  before the first included header.  However, do not add this to header
  files, esp. not to header files that are installed later.
  (Note that end-users of scafacos-fcs like IMD should instead include
  an installed header like fcs.h.)

- For preprocessed Fortran source file in a method, add
    #ifdef HAVE_FCONFIG_H
    #include <fconfig.h>
    #endif

  before the first included header.

- In the toplevel fconfig.h.in file, add the following lines for your solver:
    ! Whether solver method <solver> is enabled.
    #undef FCS_ENABLE_<SOLVER>

- Ensure src and test are adjusted (FIXME: please elaborate here)

- Finally, rerun ./bootstrap in the toplevel source directory, then proceed
  as described in README.
\end{verbatim}
%%% Local Variables: 
%%% mode: latex
%%% TeX-master: ug.tex
%%% End: 
