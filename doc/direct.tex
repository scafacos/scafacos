\chapter{direct -- Direct summation}
\label{cha:direct}

\solvertoindex{direct}
\solvertoindex{Direct Summation}

The direct solver implemented within the ScaFaCoS library performs a direct summation of the pair-wise interactions between all given particles.
The parallel computations use the given distribution of particles among parallel MPI processes (i.e., no additional redistribution for improving the load-balancing is performed).
Let $p$ be the number of parallel MPI processes.
After each process has computed the interactions between its local particles, $p-1$ steps are performed to compute interactions with all non-local particles.
In each step, each process receives the particles of the preceding process, computes the interactions with its local particles, and sends to the previously received particles to the succeeding process.

\section*{Common capabilities}

\begin{description}

  \item[Periodicity:]
Open, periodic, or mixed boundaries are supported.
The box shape is not limited by the chosen periodicity.
Periodic boundaries are computed by placing a number of periodic images of the particle system around the given particle system.
The number of images to use in each (periodic) dimensions can be specified with \texttt{fcs\_direct\_set\_periodic\_images}.

  \item[Box shape:] Any (triclinic) box shape is supported.
  
  \item[Tolerances:] No.

  \item[Delegate near-field:] No.

  \item[Virial:] ?

\end{description}

\section*{Additional capabilities}

\begin{description}

  \item[Cutoff:]
\texttt{fcs\_direct\_set\_cutoff} can be used to specify a cutoff range that limits the computation of interactions.
If the cutoff range is 0, then all interactions are considered.
If the cutoff range is greater than 0, then only interactions inside the cufoff range are considered.
If the cutoff range is less than 0, then only interactions outside the (positive) cufoff range are considered.
If the cutoff range is greater than 0, then \texttt{fcs\_direct\_set\_cutoff\_with\_near} can be used to enable the ScaFaCoS internal near-field solver module instead of the direct solver.

\end{description}

\section*{Solver specific functions}

\begin{itemize}

  \item
\begin{alltt}
fcs\_direct\_set\_cutoff(FCS handle, fcs_float cutoff);
\end{alltt}
Set the current cutoff range (optional, default = 0).

  \item
\begin{alltt}
fcs_direct_get_cutoff(FCS handle, fcs_float *cutoff);
\end{alltt}
Retrieve the current cutoff range.

  \item
\begin{alltt}
fcs_direct_set_periodic_images(FCS handle, fcs_int *periodic_images);
\end{alltt}
Set the number of periodic images to use for computations with periodic boundaries (optional, default = \{ 1, 1, 1 \}).

  \item
\begin{alltt}
fcs_direct_get_periodic_images(FCS handle, fcs_int *periodic_images);
\end{alltt}
Retrieve the number of periodic images used for computations with periodic boundaries.

  \item
\begin{alltt}
fcs_direct_set_cutoff_with_near(FCS handle, fcs_bool cutoff_with_near);
\end{alltt}
Enable the near-field solver module (instead of the direct solver) to be used for computations with a cutoff range.

  \item
\begin{alltt}
fcs_direct_get_cutoff_with_near(FCS handle, fcs_bool *cutoff_with_near);
\end{alltt}
Retrieve whether the near-field solver module is used for computations with a cutoff range.

\end{itemize}

\section*{Known bugs or missing features}

\begin{itemize}
  \item Periodic boundaries still need to be tested.
  \item How to compute the virial for periodic boundaries?
\end{itemize}

%%% Local Variables: 
%%% mode: latex
%%% TeX-master: ug.tex
%%% End: 
